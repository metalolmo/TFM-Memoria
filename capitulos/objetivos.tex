%%%%%%%%%%%%%%%%%%%%%%%%%%%%%%%%%%%%%%%%%%%%%%%%%%%%%%%%%%%%%%%%%%%%%%%%
% Plantilla TFG/TFM
% Escuela Politécnica Superior de la Universidad de Alicante
% Realizado por: Jose Manuel Requena Plens
% Contacto: info@jmrplens.com / Telegram:@jmrplens
%%%%%%%%%%%%%%%%%%%%%%%%%%%%%%%%%%%%%%%%%%%%%%%%%%%%%%%%%%%%%%%%%%%%%%%%

\chapter{Objetivos}
\label{objetivos}

En este capítulo se definen los objetivos que guían el desarrollo del presente trabajo. Estos objetivos se han clasificado en principal, secundarios y transversales, con el fin delimitar claramente el alcance del proyecto.

\section{Objetivo principal}

El objetivo principal del trabajo es desarrollar un sistema web integral de monitorización de red con capacidad de detectar anomalías en su funcionamiento. Este sistema permitirá obtener información en tiempo real sobre el estado de los routers y sus interfaces, procesarla de forma eficiente y presentarla en una interfaz gráfica amigable para el usuario final.

\section{Objetivos secundarios}

De ese objetivo principal se derivan los siguientes objetivos secundarios:

\begin{itemize}
  \item Diseñar e implementar un backend basado en FastAPI que permita la comunicación con los dispositivos de red mediantes conexiones SSH.
  \item Desarrollar un frontend web con Vue.js, Bootstrap y Chart.js para la visualización de datos en tiempo real, incluyendo el estado de interfaces, tráfico y errores detectados.
  \item Implementar un mecanismo de detección de anomalías (por ejemplo, mediante correo electrónico con informes en PDF) cuando se detecten incidencias relevantes en la red.
  \item Garantizar la seguridad y control de acceso al sistema mediante autenticación de usuarios. 
\end{itemize}

\section{Objetivos transversales}

Además, el proyecto contribuye al desarollo de una serie de competencias transversales:

\begin{itemize}
  \item Capacidad de investigación y aprendizaje autónomo. Adquisición de conocimientos avanzados sobre protocolos de red, monitorización y tecnologías web.
  \item Integración de diferentes areas de conocimiento. Combinación de ingeniería de telecomunicación, programación web, seguridad informática y administración de redes.
  \item Aplicación práctica de herramientas profesionales. Uso de tecnologías modernas (FastAPI, Vue.js, Bootstrap, Chart.js) y entornos de virtualización/laboratorio para pruebas.
  \item Trabajo orientado a la innovación. Desarrollo de un sistema con valor añadido mediante la detección de anomalías, más allá de la mera monitorización estática.
\end{itemize}  