%%%%%%%%%%%%%%%%%%%%%%%%%%%%%%%%%%%%%%%%%%%%%%%%%%%%%%%%%%%%%%%%%%%%%%%%
% Plantilla TFG/TFM
% Escuela Politécnica Superior de la Universidad de Alicante
% Realizado por: Jose Manuel Requena Plens
% Contacto: info@jmrplens.com / Telegram:@jmrplens
%%%%%%%%%%%%%%%%%%%%%%%%%%%%%%%%%%%%%%%%%%%%%%%%%%%%%%%%%%%%%%%%%%%%%%%%

\chapter{Introducción}

\section{Motivación}

La gestión de redes de telecomunicaciones se enfrenta a un escenario cada vez más complejo, en el que la administración manual de dispositivos resulta ineficiente y propensa a errores. La creciente demanda de disponibilidad y rapidez en la resolución de incidencias ha impulsado la necesidad de automatizar procesos de monitorización y configuración. En este contexto, el uso de APIs para interactuar con dispositivos de red se ha convertido en un elemento clave para agilizar tareas de supervisión, mejorar la interoperabilidad y reducir los tiempos de respuesta de los administradores.

La motivación de este proyecto surge precisamente de esta necesidad: desarrollar un sistema que integre tecnologías web y APIs para obtener información en tiempo real de los routers y automatizar tanto la supervisión como ciertas operaciones de gestión. Tal y como indican trabajos recientes en la literatura, la incorporación de automatización y monitorización mejora la capacidad de respuesta y refuerza la fiabilidad de las infraestructuras \citep{Schummer2024}.