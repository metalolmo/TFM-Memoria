% !TeX program = xelatex
% !TeX TXS-program:compile = txs:///xelatex/[--shell-escape]
%%%%%%%%%%%%%%%%%%%%%%%%%%%%%%%%%%%%%%%%%%%%%%%%%%%%%%%%%%%%%%%%%%%%%%%%
% Plantilla TFG/TFM
% Escuela Politécnica Superior de la Universidad de Alicante
% Realizado por: Jose Manuel Requena Plens
% Contacto: info@jmrplens.com / Telegram:@jmrplens
%%%%%%%%%%%%%%%%%%%%%%%%%%%%%%%%%%%%%%%%%%%%%%%%%%%%%%%%%%%%%%%%%%%%%%%%

% Elige si deseas optimizar la ejecución del proyecto almacenando las figuras generadas con TikZ y PGF en una carpeta (archivos/figuras-procesadas).
% 1 - Si, 2 - No
\def\OptimizaTikZ{1}

% Archivo .TEX que incluye todas las configuraciones del documento y los paquetes. Añade todo aquello que necesites utilizar en el documento en este archivo.
% En él se encuentra la configuración de los márgenes, establecidos según las directrices de estilo de la EPS.
\input{include/configuracioninicial}

%%%%%%%%%%%%%%%%%%%%%%%%%%%%%%%%%%%%%%%%%%%%%%%%%%%%%%%%%%%%%%%%%%%%%%
% INFORMACIÓN DEL TFG
% Comentar lo que NO se desee añadir y sustituir con la información correcta.
%%%%%%%%%%%%%%%%%%%%%%%%%%%%%%%%%%%%%%%%%%%%%%%%%%%%%%%%%%%%%%%%%%%%%%
% Título y subtítulo
\newcommand{\titulo}{Sistema vía Web de monitorización de red con detección de anomalías}
\newcommand{\subtitulo}{Implementación de un sistema web de supervisión y alerta de red, probado en entornos simulados con GNS3}
% Datos del autor
\newcommand{\miNombre}{Francisco José Olmo Valverde}
% Determinar género para etiquetas Autore/Autora/Autor (nb o en blanco,f,m)
\newcommand{\miGenero}{m}
\newcommand{\miEmail}{fjov1@alu.ua.es}
% Datos del tutor/es
% Si no hay tutorB, comentar tutorB y dptoB para que la etiqueta sea Tutor:
\newcommand{\miTutor}{Jaume Aragonés Ferrero}
%\newcommand{\miTutorB}{Nombre Apellido1 Apellido2 (tutor2)}
\newcommand{\departamentoTutor}{Lenguajes y Sistemas Informáticos}
%\newcommand{\departamentoTutorB}{Departamento del cotutor}
% Datos de la facultad y universidad
\newcommand{\miFacultad}{Escuela Politécnica Superior}
\newcommand{\miFacultadCorto}{EPS UA}
\newcommand{\miUniversidad}{\protect{Universidad de Alicante}}
\newcommand{\miUbicacion}{Alicante}

%%%%%%%%%%%%%%%%%%%%%%%%%%%%%%%%%%%%%%%%%%%%%%%%%%%%%%%%%%%%%%%%%%%%%%
% INDICA TU TITULACIÓN
% ID	GRADO -------------------------------------------------
% 1		Ingeniería en Imagen y Sonido en Telecomunicación
% 2		Ingeniería Civil
% 3		Ingeniería Química
% 4		Ingeniería Informática
% 5		Ingeniería Multimedia
% 6		Arquitectura Técnica
% 7		Arquitectura
% 8		Robótica
% %		%%%%%%%%%%%%
% ID	MÁSTER ------------------------------------------------
% A		Telecomunicación
% B		Caminos, Canales y Puertos
% C		Gestión en la Edificación
% D		Desarrollo Web
% E		Materiales, Agua, Terreno
% F		Informática
% G 	Automática y Robótica
% H		Prevención de riesgos laborales
% I		Gestión Sostenible Agua
% J		Desarrollo Aplicaciones Móviles
% K		Ingeniería Química
% L		Ciberseguridad
% M		Ingeniería Geológica
%%%%%%%%%%%%%%%%%%%%%%%%%%%%%%%%%%%%%%%%%%%%%%%%%%%%%%%%%%%%%%%%%%%%%%%%%
%!!!!!!!!!!!!!!!!!!!!!!!!!!!!!!!!!!!!!!!!!!!!!!!!!!!!!!!!!!!!!!!!!!!!!%%%
																		%
\def\IDtitulo{A} % INTRODUCE LA ID DE TU TITULACIÓN						%
																		%
%!!!!!!!!!!!!!!!!!!!!!!!!!!!!!!!!!!!!!!!!!!!!!!!!!!!!!!!!!!!!!!!!!!!!!%%%
%%%%%%%%%%%%%%%%%%%%%%%%%%%%%%%%%%%%%%%%%%%%%%%%%%%%%%%%%%%%%%%%%%%%%%%%%

% Configuración automática según el identificador elegido
\input{include/configuraciontitulacion} 

% Información añadida a las propiedades del archivo PDF.
\usepackage{pdflscape}
\hypersetup{
pdfauthor = {\miNombre~(\miEmail)},
pdftitle = {\titulo},
}

%%
% Archivo de acrónimos
%%
\makeglossaries % Genera la base de datos de acrónimos
\input{anexos/acronimos.tex} % Archivo que contiene los acrónimos

%%%%%%%%%%%%%%%%%%%%%%%% 
% INICIO DEL DOCUMENTO
% A partir de aquí debes empezar a realizar tu TFG/TFM
%%%%%%%%%%%%%%%%%%%%%%%%
\usepackage{apacite}
\usepackage[hidelinks]{hyperref} % opcional
\begin{document}

% Números romanos hasta el mainmatter.
\frontmatter

% PORTADA
\input{include/portada/portada_color} % Portada Color
\input{include/portada/portada_bn} % Portada B/N

%%%%% PREAMBULO
% Incluye el .tex que contiene el preámbulo, agradecimientos y dedicatorias.
%%%%%%%%%%%%%%%%%%%%%%%%%%%%%%%%%%%%%%%%%%%%%%%%%%%%%%%%%%%%%%%%%%%%%%%%
% Plantilla TFG/TFM
% Escuela Politécnica Superior de la Universidad de Alicante
% Realizado por: Jose Manuel Requena Plens
% Contacto: info@jmrplens.com / Telegram:@jmrplens
%%%%%%%%%%%%%%%%%%%%%%%%%%%%%%%%%%%%%%%%%%%%%%%%%%%%%%%%%%%%%%%%%%%%%%%%

\chapter{Resumen}
En la actualidad, la infraestruturas de red presentan un grado creciente de complejidad debido a la diversidad de dispositivos y servicios que las componen. Esta situación exige herramientas avanzadas que permitan supervisar el estado de la red en tiempo real, detectar incidencias de manera temprana y facilitar la toma de decisiones por parte de los administradores.

El siguiente Trabajo Final de Máster se aborda el diseño e implementación de un sistema web de monitorización de red con capacidad de detección de anomalías. El objetivo principal es una plataforma integral que permita obtener la información relevante de los dispositivos de red, analizar su comportamiento y representar los resultados en una interfaz visual clara e interactiva.

La solución se ha desarrollado siguiendo una arquitectura cliente-servidor. El backend, implementado en Python mediante el framework FastAPI, establece conexión con router Cisco del emulador GNS3 utilizando el protocolo SSH (a través de la librería Paramiko) para extraer métricas como el estado de las interfaces, tráfico de entrada y salida, utilización de CPU y memoria, así como cambios de configuración en los dispositivos. Por su parte, el frontend se ha construido con Vue.js Bootstrap y Chart.js, ofreciendo visualización dinámica de los datos mediante gráficos en tiempo real, tablas interactivas y barras de progreso que facilitan la interpretación de los parámetros monitorizados.

Además, el sistema incorpora mecanismo de seguridad y gestión de usuarios mediante autenticación basada en JSON Web Tokens (JWT) Lo que garantiza el acceso restringido a la aplicación. También se ha implementado un módulo de generación automática de informes en formato PDF y su envío por correo electrónico en caso de detección de errores.

Los resultados obtenidos evidencia que la solución propuesta cumple los objetivos planteados: permite monitorizar en tiempo real distintos dispositivos de red, detectar anomalías en su funcionamiento y presentar la información de forma accesible y comprensible. Este sistema constituye una herramienta de apoyo útil para la gestión de redes de telecomunicaciones, con potencial de apliación hacia entornos más complejos. 
%%%%%%%%%%%%%%%%%%%%%%%%%%%%%%%%%%%%%%%%%%%%%%%%%%%%%%%%%%%%%%%%%%%%%%%%
% Plantilla TFG/TFM
% Escuela Politécnica Superior de la Universidad de Alicante
% Realizado por: Jose Manuel Requena Plens
% Contacto: info@jmrplens.com / Telegram:@jmrplens
%%%%%%%%%%%%%%%%%%%%%%%%%%%%%%%%%%%%%%%%%%%%%%%%%%%%%%%%%%%%%%%%%%%%%%%%

\chapter{Abstract}
In today's context, network infrastructures present an increasing degree of complexity due to the diversity of devices and services they encompass. This situation requires advanced tools capable of continuously supervising the network status in real time, detecting incidents at an early stage, and facilitating decision-making for administrators.  

This Master’s Thesis addresses the design and implementation of a web-based network monitoring system with anomaly detection capabilities. The main objective is to provide an integral platform that allows the retrieval of relevant information from network devices, the analysis of their behavior, and the representation of results through a clear and interactive visual interface.  

The solution has been developed following a client-server architecture. The backend, implemented in Python using the FastAPI framework, connects to Cisco routers in the GNS3 emulator via the SSH protocol (through the Paramiko library) to extract metrics such as interface status, inbound and outbound traffic, CPU and memory utilization, as well as configuration changes in the devices. The frontend, developed with Vue.js, Bootstrap, and Chart.js, offers a dynamic visualization of the data through real-time charts, interactive tables, and progress bars that facilitate the interpretation of monitored parameters.  

In addition, the system integrates security mechanisms and user management through JSON Web Token (JWT) authentication, ensuring restricted access to the application. An automatic report generation module in PDF format and its delivery via email in case of error detection has also been implemented.  

The results obtained demonstrate that the proposed solution meets the established objectives: it enables real-time monitoring of different network devices, detects anomalies in their operation, and presents the information in an accessible and understandable manner. This system constitutes a valuable support tool for the management of telecommunication networks, with potential for extension to more complex environments.
 
\input{capitulos/preliminaresconagradecimientos} 

% Incluye después del archivo anterior el indice y lista de figuras, tablas y códigos.
\tableofcontents	% Índice
\listoffigures		% Índice de figuras
\listoftables		% Índice de tablas
\lstlistoflistings	% Índice de códigos

% Inicia la numeración habitual.
\mainmatter

%%%%
% CONTENIDO. CAPÍTULOS DEL TRABAJO - Añade o elimina según tus necesidades
%%%%
%%%%%%%%%%%%%%%%%%%%%%%%%%%%%%%%%%%%%%%%%%%%%%%%%%%%%%%%%%%%%%%%%%%%%%%%
% Plantilla TFG/TFM
% Escuela Politécnica Superior de la Universidad de Alicante
% Realizado por: Jose Manuel Requena Plens
% Contacto: info@jmrplens.com / Telegram:@jmrplens
%%%%%%%%%%%%%%%%%%%%%%%%%%%%%%%%%%%%%%%%%%%%%%%%%%%%%%%%%%%%%%%%%%%%%%%%

\chapter{Introducción}

\section{Motivación}

La gestión de redes de telecomunicaciones se enfrenta a un escenario cada vez más complejo, en el que la administración manual de dispositivos resulta ineficiente y propensa a errores. La creciente demanda de disponibilidad y rapidez en la resolución de incidencias ha impulsado la necesidad de automatizar procesos de monitorización y configuración. En este contexto, el uso de APIs para interactuar con dispositivos de red se ha convertido en un elemento clave para agilizar tareas de supervisión, mejorar la interoperabilidad y reducir los tiempos de respuesta de los administradores.

La motivación de este proyecto surge precisamente de esta necesidad: desarrollar un sistema que integre tecnologías web y APIs para obtener información en tiempo real de los routers y automatizar tanto la supervisión como ciertas operaciones de gestión. Tal y como indican trabajos recientes en la literatura, la incorporación de automatización y monitorización mejora la capacidad de respuesta y refuerza la fiabilidad de las infraestructuras \citep{Schummer2024}.	% Plantilla: Se muestran contenidos
%%%%%%%%%%%%%%%%%%%%%%%%%%%%%%%%%%%%%%%%%%%%%%%%%%%%%%%%%%%%%%%%%%%%%%%%
% Plantilla TFG/TFM
% Escuela Politécnica Superior de la Universidad de Alicante
% Realizado por: Jose Manuel Requena Plens
% Contacto: info@jmrplens.com / Telegram:@jmrplens
%%%%%%%%%%%%%%%%%%%%%%%%%%%%%%%%%%%%%%%%%%%%%%%%%%%%%%%%%%%%%%%%%%%%%%%%

\chapter{Estudio de Viabilidad}
\section{Análisis DAFO}

El análisis DAFO realizado en este trajo se centra en evaluar los factores internos y externos que influyen en el desarrollo del sistema web.

Dentro de las \textbf{fortalezas}, se destaca el uso de stack tecnológico moderno y eficiente (FastAPI en el backend, Vue y Bootstrap en el frotend y Chart.js para la representación gráfica), lo que permite un desarrollo modular rápido y con gran capacidad de visualización en tiempo real. Asimismo, la integración mediante conexiones SSH a routers Cisco otorga al sistema una aplicación práctica inmediata.

En cuanto a las \textbf{debilidades}, se identifican la experiencia limitada en el uso de algoritmos avanzados de detección de anomalías, la dependencia inicial de un único fabricante (Cisco) y la necesidad de mayor tiempo y recursos para ampliar la solución hacia entornos más complejos.

Respecto a las \textbf{oportunidades}, el sistema se orienta principalmente principalmente a un entorno académico, donde puede servir como apoyo al aprendizaje de conceptos avanzados de redes y monitorización. Esta herramienta facilita la compresión práctica de la interpretación de métricas de tráfico y la detección de anomalías, aspectos que resultan especialmente útiles en ingeniería de telecomunicación. Asimismo, puede constituir a una base sobre la cual los estudiantes experimenten con nuevas técnicas de análisis.

Por último, entre las \textbf{amenazas} se encuentran la rápida evolución de las ciberamenazas y las soluciones comerciales ya consolidadas en el mercado (Nagios, Zabbix, Prometheus, entre otras),que pueden limitar la implantación del sistema en entornos productivos. Además, los posibles cambios en las interfaces de configuración de los fabricantes podrían afectar a la compatibilidad a largo plazo.

La Figura~\ref{fig:dafo} muestra de forma esquemática este análisis aplicado
al proyecto desarrollado.

\begin{landscape}
\begin{figure}[H]
  \centering
  \includegraphics[height=\textwidth]{canva/DAFO.pdf}
  \caption{Análisis DAFO del proyecto}
  \label{fig:dafo}
\end{figure}
\end{landscape}	% Plantilla: Se muestran listas
%%%%%%%%%%%%%%%%%%%%%%%%%%%%%%%%%%%%%%%%%%%%%%%%%%%%%%%%%%%%%%%%%%%%%%%%
% Plantilla TFG/TFM
% Escuela Politécnica Superior de la Universidad de Alicante
% Realizado por: Jose Manuel Requena Plens
% Contacto: info@jmrplens.com / Telegram:@jmrplens
%%%%%%%%%%%%%%%%%%%%%%%%%%%%%%%%%%%%%%%%%%%%%%%%%%%%%%%%%%%%%%%%%%%%%%%%

\chapter{Objetivos}
\label{objetivos}

En este capítulo se definen los objetivos que guían el desarrollo del presente trabajo. Estos objetivos se han clasificado en principal, secundarios y transversales, con el fin delimitar claramente el alcance del proyecto.

\section{Objetivo principal}

El objetivo principal del trabajo es desarrollar un sistema web integral de monitorización de red con capacidad de detectar anomalías en su funcionamiento. Este sistema permitirá obtener información en tiempo real sobre el estado de los routers y sus interfaces, procesarla de forma eficiente y presentarla en una interfaz gráfica amigable para el usuario final.

\section{Objetivos secundarios}

De ese objetivo principal se derivan los siguientes objetivos secundarios:

\begin{itemize}
  \item Diseñar e implementar un backend basado en FastAPI que permita la comunicación con los dispositivos de red mediantes conexiones SSH.
  \item Desarrollar un frontend web con Vue.js, Bootstrap y Chart.js para la visualización de datos en tiempo real, incluyendo el estado de interfaces, tráfico y errores detectados.
  \item Implementar un mecanismo de detección de anomalías (por ejemplo, mediante correo electrónico con informes en PDF) cuando se detecten incidencias relevantes en la red.
  \item Garantizar la seguridad y control de acceso al sistema mediante autenticación de usuarios. 
\end{itemize}

\section{Objetivos transversales}

Además, el proyecto contribuye al desarollo de una serie de competencias transversales:

\begin{itemize}
  \item Capacidad de investigación y aprendizaje autónomo. Adquisición de conocimientos avanzados sobre protocolos de red, monitorización y tecnologías web.
  \item Integración de diferentes areas de conocimiento. Combinación de ingeniería de telecomunicación, programación web, seguridad informática y administración de redes.
  \item Aplicación práctica de herramientas profesionales. Uso de tecnologías modernas (FastAPI, Vue.js, Bootstrap, Chart.js) y entornos de virtualización/laboratorio para pruebas.
  \item Trabajo orientado a la innovación. Desarrollo de un sistema con valor añadido mediante la detección de anomalías, más allá de la mera monitorización estática.
\end{itemize}  		% Plantilla: Se muestran tablas
%%%%%%%%%%%%%%%%%%%%%%%%%%%%%%%%%%%%%%%%%%%%%%%%%%%%%%%%%%%%%%%%%%%%%%%%
% Plantilla TFG/TFM
% Escuela Politécnica Superior de la Universidad de Alicante
% Realizado por: Jose Manuel Requena Plens
% Contacto: info@jmrplens.com / Telegram:@jmrplens
%%%%%%%%%%%%%%%%%%%%%%%%%%%%%%%%%%%%%%%%%%%%%%%%%%%%%%%%%%%%%%%%%%%%%%%%

\chapter{Estado del Arte}

En este capítulo se presentan las principales soluciones existentes en el ámbito de la monitorización de redes. El objetivo es identificar sus características más relevantes, así como sus ventajas y limitaciones, de manera que sirvan como referencia para el diseño del sistema propuesto en este trabajo.

\section{Nagios: Sistema de Monitorización de Redes}
Nagios es una de las soluciones más utilizadas en el ámbito de la monitorización de infraestructuras IT. Su capacidad de supervisar redes, servidores y aplicaciones la convierten en una opción versátil y ampliamente adoptada en entornos empresariales. 

Uno de los principales productos de Nagios es Nagios XI, el cual proporciona un conjunto de herramientas avanzadas para la monitorización de infraestructuras críticas \citep{nagios2014}. Este software perminte supervisar el estado de aplicaciones, servicios, sistemas operativos y protocolos de red. Además, cuenta con un sistema de alertas en tiempo real que notifica a los administradores mediante correo electrónico, SMS o mensajería instantánea. Tiene la capacidad de generar informes detallados sobre el historial de fallos, análisis de tendencias y planificación de capacidad es otra de sus características más destacadas. Una de las fortalezas es su interfaz web, la cual permite a los usuarios personalizar paneles de control para visualizar métricas clave \citep{nagios2014}.

\begin{figure}[H]
    \centering
    {\includegraphics[width=0.7\linewidth]{imagenes/image2.png}}
    \caption{Ejemplo visual de Nagios XI \citep{nagios2014}}
    \label{fig:enter-label}
\end{figure}

Otro componente relevante es Nagio Fusion, diseñado para proporcionar una vista consolidad de múltiples instancias de Nagios \citep{nagios2014}. A gran escala, esta herramienta resulta especialmente útil, ya que permite gestionar diferentes servidores de monitoreo desde una única plataforma \citep{nagios2014}.

\begin{figure}[H]
    \centering
    {\includegraphics[width=0.7\linewidth]{imagenes/image3.png}}
    \caption{Ejemplo visual de Nagios Fusion \citep{nagios2014}.}
    \label{fig:enter-label}
\end{figure}

En el ámbito de la gestión de incidentes, Nagios Incident Manager facilita  la administración de alertas y eventos críticos detectados por Nagios XI \citep{nagios2014}. Su integración permite la creación y la asignación de tickets en tiempo real, fomentando la colaboración entre equipos IT y agilizando la resolución de problemas.

\begin{figure}[H]
    \centering
    {\includegraphics[width=0.7\linewidth]{imagenes/image4.png}}
    \caption{Ejemplo visual de Nagios Incident Manager \citep{nagios2014}.}
    \label{fig:enter-label}
\end{figure}

Por otro lado, Nagios Network Analyzer se enfoca en el análisis de tráfico de red \citep{nagios2014}. Esta herramienta permite supervisar el uso del ancho de banda y detectar patrones de tráfico que pueden indicar problemas o posibles amenazas de seguridad. Su compatibilidad con tecnologías como NetFlow y sFlow permite recopilar información detallada sobre el tráfico en la red, alertando a los administradores en caso de picos inusuales o comportamientos anómalos. Además, su capacidad de generar gráficos avanzados facilita la interpretación de datos y el diagnóstico de problemas en la red \citep{nagios2014}.

\begin{figure}[H]
    \centering
    {\includegraphics[width=0.7\linewidth]{imagenes/image5.png}}
    \caption{Ejemplo Visual de Nagios Network Analyzer \citep{nagios2014}.}
    \label{fig:enter-label}
\end{figure}

Dentro del contexto del Trabajo final de Máster, Nagios representa un modelo robusto que puede servir de referencia. En particular, Nagios XI y Nagios Network Analyzer, la posibilidad de personalizar dashboards, gestionar alertas en tiempo real y analizar tráfico de red son clave en la planificación y el diseño del trabajo.

Sin embargo, también es importante indicar las limitaciones del software. Aunque su sistema de monitorización es configurable, puede requerir una curva de aprendizaje elevada, especialmente para administradores sin experiencia previa. Además, su modelo basado en plugins y extensiones puede hacer que la integración de otros sitemas requiera una configuración adicional.
% Aquí copiarías lo que ya tienes desarrollado sobre Nagios (texto + figuras).

\section{Zabbix: Sistema de Monitorización de Redes}
Zabbix es una plataforma de monitorización de código abierto que ha sido ampliamente adoptada en la industria debido a su flexibilidad, escalabilidad y capacidad de integración con diversas tecnologías.

% Aquí copiarías el bloque de Zabbix.

\section{Prometheus: Sistema de Monitorización de Redes}
Prometheus es una plataforma de monitorización y alerta de código abierto diseñada para la recolección y análisis de métricas en entornos dinámicos.

% Aquí copiarías lo de Prometheus.

\section{SolarWinds Network Performance Monitor}
SolarWinds Network Performance Monitor (NPM) es una solución comercial diseñada para proporcionar una visión integral del estado y rendimiento de infraestructuras de redes de comunicaciones.

% Aquí copias lo de SolarWinds.

\section{PRTG Network Monitor}
PRTG Network Monitor, desarrollado por Paessler AG, es una plataforma de monitorización de redes que destaca por su enfoque integral, intuitivo y escalable.

% Aquí copias lo de PRTG.

\section{Comparativa de herramientas}
Con el fin de obtener una visión global, en la Figura~\ref{fig:comparativa} se presenta una tabla comparativa que sintetiza las características principales de las herramientas analizadas.

\begin{figure}[H]
    \centering
    \includegraphics[width=\textwidth]{imagenes/image13.png}
    \caption{Tabla comparativa con los sistemas de monitorización analizados.}
    \label{fig:comparativa}
\end{figure}

\section{Conclusiones del estado del arte}
Del análisis realizado se desprende que:
\begin{itemize}
    \item \textbf{Nagios} y \textbf{Zabbix} son soluciones robustas y muy extendidas, aunque pueden requerir una curva de aprendizaje elevada.
    \item \textbf{Prometheus}, en combinación con Grafana, se ha consolidado como la opción preferida en entornos de microservicios y Kubernetes.
    \item \textbf{SolarWinds} y \textbf{PRTG} ofrecen soluciones comerciales completas, con interfaces gráficas avanzadas, pero requieren licencias de pago.
    \item Todas las herramientas incorporan mecanismos de alertas y notificaciones, siendo este un elemento esencial en la gestión proactiva de redes.
\end{itemize}

En conclusión, cada herramienta presenta fortalezas y debilidades, lo que evidencia la necesidad de un sistema que combine escalabilidad, facilidad de uso y capacidad de detección de anomalías. Este enfoque será el que guíe el desarrollo del presente TFM.	% Plantilla: Se muestran figuras
\input{capitulos/metodologia y planificacion}		% Plantilla: Se muestran listados
\input{capitulos/analisis}		% Plantilla: Se muestran gráficas
\input{capitulos/diseño}	% Plantilla: Se muestran matemáticas
\input{capitulos/implementacion}	% Plantilla: Se muestran matemáticas
\input{capitulos/pruebas}	% Plantilla: Se muestran matemáticas
\input{capitulos/resultados}	% Plantilla: Se muestran matemáticas
\input{capitulos/conclusiones}	% Plantilla: Se muestran matemáticas
\input{capitulos/trabajo futuro}	% Plantilla: Se muestran matemáticas

%%%%
% CONTENIDO. BIBLIOGRAFÍA.
%%%%
\nocite{*} %incluye TODOS los documentos de la base de datos bibliográfica sean o no citados en el texto
\bibliographystyle{apacite}
\bibliography{bibliografia/bibliografia} % Archivo que contiene la bibliografía


%%%%
% CONTENIDO. LISTA DE ACRÓNIMOS. Comenta las líneas si no lo deseas incluir.
%%%%
% Incluye el listado de acrónimos utilizados en el trabajo. 
\printglossary[style=modsuper,type=\acronymtype,title={Lista de Acrónimos y Abreviaturas}]
% Añade el resto de acrónimos si así se desea. Si no elimina el comando siguiente
\glsaddallunused 

%%%%
% CONTENIDO. Anexos - Añade o elimina según tus necesidades
%%%%
\appendix % Inicio de los apéndices
\input{anexos/anexo_I}
%%%%%%%%%%%%%%%%%%%%%%%%%%%%%%%%%%%%%%%%%%%%%%%%%%%%%%%%%%%%%%%%%%%%%%%%
% Plantilla TFG/TFM
% Escuela Politécnica Superior de la Universidad de Alicante
% Realizado por: Jose Manuel Requena Plens
% Contacto: info@jmrplens.com / Telegram:@jmrplens
%%%%%%%%%%%%%%%%%%%%%%%%%%%%%%%%%%%%%%%%%%%%%%%%%%%%%%%%%%%%%%%%%%%%%%%%


% Ejemplo de páginas en horizontal y vertical

\chapter{Páginas horizontales}



%\input{anexos/anexo_3}

\end{document}
